\documentclass[a4paper]{article}
\usepackage[utf8]{inputenc}
\usepackage[T1]{fontenc}
\usepackage[catalan]{babel}
\usepackage{amsmath, amsthm, amssymb}
\usepackage{lmodern}
\usepackage{microtype}

\newcommand{\ZZ}{\mathbb{Z}}

\DeclareMathOperator{\sgn}{sgn}

\newenvironment{solution}{
    \renewcommand\qedsymbol{\ensuremath{\lozenge}}
    \begin{proof}[Solució]
        }{
    \end{proof}
}

\title{Símbol de Jacobi}
\author{Claudi Lleyda Moltó}
\date{1462908}

\begin{document}
\maketitle

\paragraph{El Símbol de Jacobi} Sigui~\(m\) un enter positiu senar.
Podem escriure així~\(m=p_{1}\cdots p_{s}\) on els~\(p_{i}\) són primers senars,
no necessàriament diferents.
Es defineix el~\emph{símbol de Jacobi} com:
\[
    \Bigl(\frac{a}{m}\Bigr)
    =
    \prod_{i=1}^{s}
    \Bigl(\frac{a}{p_{i}}\Bigr).
\]

\begin{enumerate}
    \item[\textbf{a)}] Sigui~\(m\geq2\) un nombre enter.
        Donat~\(a\in\ZZ\) un nombre primer amb~\(m\),
        considerem l'aplicació~\(\mu_{a}:\ZZ/m\ZZ\longrightarrow\ZZ/m\ZZ\)
        definida com~\(\mu_{a}(n)=an\) per a tot~\(n\in\ZZ/m\ZZ\).
        \begin{enumerate}
            \item[\textbf{i)}] Demostreu que~\(\mu_{a}\) és un automorfisme.
        \end{enumerate}
\end{enumerate}

\begin{solution}
Primer veiem que~\(\mu_{a}\) és un morfisme de grups.
Per a tots~\(x,y\in\ZZ/m\ZZ\) tenim
\[
    \mu_{a}(x+y) = a(x+y) = ax + ay = \mu_{a}(x) + \mu_{a}(y)
    \qquad
    \text{i}
    \qquad
    \mu_{a}(0) = a0 = 0.
\]
Per últim veiem que per a tot~\(n\in\ZZ/m\ZZ\) es satisfà
\[
    \mu_{a}(-n) = a(-n) = -(an) = -\mu_{a}(n),
\]
i concloem que~\(\mu_{a}\) és un morfisme de grups.

Veiem ara que~\(\mu_{a}\) és bijectiva.
Per això en tenim prou amb veure que és exhaustiva,
donat que~\(\ZZ/m\ZZ\) és un finit.

Observem que podem veure que~\(\mu_{a}\) és exhaustiva veient que~\(a\) té
invers a~\(\ZZ/m\ZZ\),
ja que aleshores tindríem
\[
    \mu_{a}(a^{-1}n) = aa^{-1}n = n
\]
per a tot~\(n\in\ZZ/m\ZZ\).

Veiem doncs que~\(a\) és invertible a~\(\ZZ/m\ZZ\).
Recordem que hem suposat que~\(a\) i~\(m\) són coprimers.
Aleshores pel Lema de Bézout tenim que existeixen~\(b,c\in\ZZ\) tals que
\[
    ab + mc = 1,
\]
i per tant
\[
    ab \equiv 1 \pmod{m},
\]
i veiem~\(a\) és invertible a~\(\ZZ/m\ZZ\),
pel que tenim que~\(\mu_{a}\) és exhaustiva,
i per tant bijectiva,
com ens calia demostrar.
\end{solution}

\begin{enumerate}
    \item[] \begin{enumerate}
        \item[\textbf{ii)}] Demostreu que~\(\mu_{ab} = \mu_{a}\circ\mu_{b}\) per
            a tot~\(a\) i~\(b\) coprimers amb~\(m\).
    \end{enumerate}
\end{enumerate}

\begin{solution}
Tenim que per a tot~\(n\in\ZZ/m\ZZ\) es satisfà
\[
    \mu_{ab}(n)
    = abn
    = a(bn)
    = \mu_{a}(bn)
    = \mu_{a}\bigl(\mu_{b}(n)\bigr)
    = \mu_{a}\circ\mu_{b}(n),
\]
i per tant~\(\mu_{ab} = \mu_{a}\circ\mu_{b}\).
\end{solution}

\begin{enumerate}
    \item[] \begin{enumerate}
        \item[\textbf{iii)}] Demostreu que si~\(m=nn'\),
            i considerem el subgrup~\(H_{n}=n'\ZZ/m\ZZ\),
            aleshores tenim~\(\ZZ/n\ZZ \cong H_{n}\)
            enviant~\(b\mapsto n'b\),
            que~\(\mu_{a}(H_{n})=H_{n}\)
            i que~\(\mu_{a}\rvert_{H_{n}}=\mu_{a}\)
            via la identificació de~\(H_{n}\)
            amb~\(\ZZ/n\ZZ\).
    \end{enumerate}
\end{enumerate}

\begin{solution}
    % TODO
\end{solution}

\begin{enumerate}
    \item[] \begin{enumerate}
        \item[\textbf{iv)}] Demostreu que si~\(m=p^{r}\),
            amb~\(p\) un primer senar
            i~\(r>1\), aleshores
            \[
                (\ZZ/p^{r}\ZZ)/H_{p^{r-1}} = (\ZZ/p^{r}\ZZ)^{\times}
            \]
            és el conjunt dels elements invertibles.
            Denotem per~\(\mu_{a}^{\times}\)
            la restricció de~\(\mu_{a}\) a~\((\ZZ/p^{r}\ZZ)^{\times}\),
            i si~\(g\) és un generador de~\((\ZZ/p^{r}\ZZ)^{\times}\),
            identifiquem
            \[
                (\ZZ/p^{r}\ZZ)^{\times}
                =
                \{ g^{i} \mid i\in \{1,\dots,N\}\}
                \equiv
                \{1,\dots,N\}
            \]
            on~\(N = \lvert(\ZZ/p^{r}\ZZ)^{\times}\rvert\).
            Demostreu que, amb aquesta identificació,
            \[
                \mu_{g}^{\times} \leftrightarrow (1,2,\dots,N).
            \]
    \end{enumerate}
\end{enumerate}

\begin{solution}
    % TODO
\end{solution}

\begin{enumerate}
    \item[\textbf{b)}] Denotem que~\([\frac{a}{m}]\) el signe~\(\sgn(\mu_{a})\)
        de la permutació donada per l'aplicació~\(\mu_{a}\) a~\(\ZZ/m\ZZ\).
        \begin{enumerate}
            \item[\textbf{i)}] Demostreu que si~\(a=a_{1}a_{2}\), aleshores
                \[
                    \Bigl[\frac{a}{m}\Bigr]
                    =
                    \Bigl[\frac{a_{1}}{m}\Bigr]
                    \Bigl[\frac{a_{2}}{m}\Bigr].
                \]
        \end{enumerate}
\end{enumerate}

\begin{solution}
    Observem que si~\(a\) és coprimer amb~\(m\),~\(a_{1}\) i~\(a_{2}\) també ho
    seran. Per tant podem aplicar l'apartat~\textbf{a)ii)} per veure que
    \[
        \mu_{a} = \mu_{a_{1}a_{2}} = \mu_{a_{1}}\circ\mu_{a_{2}},
    \]
    i aleshores tenim que
    \[
        \Bigl[\frac{a}{m}\Bigr]
        = \sgn(\mu_{a})
        = \sgn(\mu_{a_{1}}\circ\mu_{a_{2}})
        = \sgn(\mu_{a_{1}})\sgn(\mu_{a_{2}})
        = \Bigl[\frac{a_{1}}{m}\Bigr]\Bigl[\frac{a_{2}}{m}\Bigr].
        \qedhere
    \]
\end{solution}

\begin{enumerate}
    \item[]\begin{enumerate}
        \item[\textbf{ii)}] Demostreu que si~\(m=m_{1}m_{2}\),
            amb~\(\gcd(m_{1},m_{2})=1\), aleshores
            \[
                \Bigl[\frac{a}{m}\Bigr]
                =
                \Bigl[\frac{a}{m_{1}}\Bigr]
                \Bigl[\frac{a}{m_{2}}\Bigr].
            \]
    \end{enumerate}
\end{enumerate}

\begin{solution}
    Podem fer la demostració per~\(m_{1},m_{2}\) primers, i generalitzar-la al
    resultat que volem, així que suposem que~\(m_{1}\) i~\(m_{2}\) són primers
    diferents. Aleshores
    \[
        \ZZ/m\ZZ
        \cong
        \ZZ/m_{1}\ZZ\times\ZZ/m_{2}\ZZ
    \]
    Observem que el signe d'una permutació és igual a~\(-1\) elevat a la meitat
    del nombre d'elements que no són fixats per aquesta. Per tant el signe de la
    permutació~\(\mu_{a}\) a~\(\ZZ/m\ZZ\) serà igual a la meitat de la quantitat
    d'elements que són fixats per~\(\mu_{a}\) simultàniament a~\(\ZZ/m_{1}\ZZ\)
    i a~\(\ZZ/m_{2}\ZZ\).

    Aleshores, si definim~\(\mu'_{1}\) com el nombre d'elements no fixats
    per~\(\mu_{a}\) a~\(\ZZ/m_{1}\ZZ\) i~\(\mu'_{2}\) com el nombre d'elements
    no fixats per~\(\mu_{a}\) a~\(\ZZ/m_{2}\ZZ\) tenim que el nombre d'elements
    no fixats per~\(\mu_{a}\) a~\(\ZZ/m\ZZ\) és la meitat de
    \[
        (m_{2}-m'_{2})m'_{1} + (m_{1}-m'_{1})m'_{2} + m'_{1}m'_{2},
    \]
    d'on veiem que aquest nombre presenta la mateixa paritat que~\(m'_{1}\)
    i~\(m'_{2}\), i hem acabat.
\end{solution}

\end{document}
