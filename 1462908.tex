\documentclass[a4paper]{article}
\usepackage[utf8]{inputenc}
\usepackage[T1]{fontenc}
\usepackage[catalan]{babel}
\usepackage{amsmath, amsthm, amssymb}
\usepackage{lmodern}
\usepackage{microtype}

\newcommand{\ZZ}{\mathbb{Z}}

\title{Símbol de Jacobi}
\author{Claudi Lleyda Moltó}
\date{1462908}

\begin{document}
\maketitle

\paragraph{El Símbol de Jacobi} Sigui~\(m\) un enter positiu senar.
Podem escriure així~\(m=p_{1}\cdots p_{s}\) on els~\(p_{i}\) són primers senars,
no necessàriament diferents.
Es defineix el~\emph{símbol de Jacobi} com:
\[
    \Bigl(\frac{a}{m}\Bigr)
    =
    \prod_{i=1}^{s}
    \Bigl(\frac{a}{p_{i}}\Bigr).
\]

\begin{enumerate}
    \item[\textbf{a)}] Sigui~\(m\geq2\) un nombre enter.
        Donat~\(a\in\ZZ\) un nombre primer amb~\(m\),
        considerem l'aplicació~\(\mu_{a}:\ZZ/m\ZZ\longrightarrow\ZZ/m\ZZ\)
        definida com~\(\mu_{a}(n)=an\) per a tot~\(n\in\ZZ/m\ZZ\).
        \begin{enumerate}
            \item[\textbf{i)}] Demostreu que~\(\mu_{a}\) és un automorfisme.
        \end{enumerate}
\end{enumerate}

Primer veiem que~\(\mu_{a}\) és un morfisme de grups.
Per a tots~\(x,y\in\ZZ/m\ZZ\) tenim
\[
    \mu_{a}(x+y) = a(x+y) = ax + ay = \mu_{a}(x) + \mu_{a}(y)
    \qquad
    \text{i}
    \qquad
    \mu_{a}(0) = a0 = 0.
\]
Per últim veiem que per a tot~\(n\in\ZZ/m\ZZ\) es satisfà
\[
    \mu_{a}(-n) = a(-n) = -(an) = -\mu_{a}(n),
\]
i concloem que~\(\mu_{a}\) és un morfisme de grups.

Veiem ara que~\(\mu_{a}\) és bijectiva.
Per això en tenim prou amb veure que és exhaustiva,
donat que~\(\ZZ/m\ZZ\) és un finit.

Observem que podem veure que~\(\mu_{a}\) és exhaustiva veient que~\(a\) té
invers a~\(\ZZ/m\ZZ\),
ja que aleshores tindríem
\[
    \mu_{a}(a^{-1}n) = aa^{-1}n = n
\]
per a tot~\(n\in\ZZ/m\ZZ\).

Veiem doncs que~\(a\) és invertible a~\(\ZZ/m\ZZ\).
Recordem que hem suposat que~\(a\) i~\(m\) són coprimers.
Aleshores pel Lema de Bézout tenim que existeixen~\(b,c\in\ZZ\) tals que
\[
    ab + mc = 1,
\]
i per tant
\[
    ab \equiv 1 \pmod{m},
\]
i veiem~\(a\) és invertible a~\(\ZZ/m\ZZ\),
pel que tenim que~\(\mu_{a}\) és exhaustiva,
i per tant bijectiva,
com ens calia demostrar.

\begin{enumerate}
    \item[] \begin{enumerate}
        \item[\textbf{ii)}] Demostreu que~\(\mu_{ab} = \mu_{a}\circ\mu_{b}\) per
            a tot~\(a\) i~\(b\) coprimers amb~\(m\).
    \end{enumerate}
\end{enumerate}

Tenim que per a tot~\(n\in\ZZ/m\ZZ\) es satisfà
\[
    \mu_{ab}(n)
    = abn
    = a(bn)
    = \mu_{a}(bn)
    = \mu_{a}\bigl(\mu_{b}(n)\bigr)
    = \mu_{a}\circ\mu_{b}(n),
\]
i per tant~\(\mu_{ab} = \mu_{a}\circ\mu_{b}\).

\end{document}
